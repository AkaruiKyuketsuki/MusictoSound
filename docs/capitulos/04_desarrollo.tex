% !TeX root = ../main.tex
% Este capitulo se actualizara automaticamente mediante GitHub Actions
% No escribas nada aqui a mano


% === AUTO-GENERATED ENTRY ===

\section{Commits del 2025-10-01}
Prueba hello world.
\subsection*{Commit 1}
Initial commit.

\section{Commits del 2025-12-04}
\subsection*{Commit 1}
Añadido el registro de cambios internos y la estructura minima de la memoria.
\subsection*{Commit 2}
Creada la estructura tipo latex para docs.
\subsection*{Commit 3}
Creacion inicial de carpetas.

\section{Avances del 2025-12-04}
\subsection*{Funcion de generación de código}
En esta fase del desarrollo, se implementa una correccion intral en la funcion encargada de la generacion de la memoria tecnica del proyecto (\texttt{generate\_memoria}). La modificacion tuvo como objetivo principal mejorar la robustez en el manejo de las interacciones con la API de OpenAI, lo que se traduce en una mayor estabilidad y fiabilidad durante el proceso de generacion automatica de contenidos.

Desde el punto de vista del diseño, se revisaron los mecanismos de gestion de errores y excepciones relacionados con las llamadas a la API, incorporando estrategias que permiten una recuperacion efectiva ante posibles fallos de comunicacion o respuestas inesperadas. Esta decision se tomó para garantizar que el sistema pueda continuar operando sin interrupciones significativas, evitando perdidas de informacion o resultados incompletos.

El impacto de estos ajustes es fundamental para asegurar una experiencia de usuario consistente y mejorar la calidad del output generado, aspectos cruciales para el exito en la presentacion y defensa del Trabajo de Fin de Grado. Ademas, este trabajo facilita futuras ampliaciones o adaptaciones, al contar con una infraestructura de manejo de API mas salida y mantenible.

El impacto de estos ajustes es fundamental para asegurar una experiencia de usuario consistente y mejorar la calidad del output generado, aspectos cruciales para el éxito en la presentación y defensa del Trabajo de Fin de Grado. Además, este trabajo facilita futuras ampliaciones o adaptaciones, al contar con una infraestructura de manejo de API más sólida y mantenible.

\subsection*{integración con la API de OpenAI}

En esta fecha se completó la integración del sistema con la API de OpenAI, lo cual representa un hito crucial en el desarrollo del proyecto. El propósito principal de este avance fue habilitar la comunicación directa entre la plataforma del trabajo y los servicios de inteligencia artificial proporcionados por OpenAI, permitiendo así el uso de modelos avanzados para el procesamiento y generación de texto.

Desde un punto de vista técnico, la decisión de enlazar con la API externa estuvo guiada por la necesidad de aprovechar capacidades de procesamiento natural del lenguaje sin incurrir en el desarrollo de modelos propios, lo que supone un ahorro significativo en recursos y tiempo. La implementación consideró aspectos clave como la gestión eficiente de las solicitudes HTTP, el manejo de respuestas asíncronas y la correcta administración de las claves de acceso para garantizar la seguridad y estabilidad del sistema.

El impacto de esta integración se refleja en la ampliación de las funcionalidades del proyecto, mejorando la calidad y precisión de las tareas automáticas que dependen de la inteligencia artificial, además de establecer una base robusta para futuras extensiones. Se realizaron pruebas exhaustivas para validar la correcta comunicación, respuesta y tolerancia a fallos en el enlace con la API, asegurando así la fiabilidad del sistema en entornos reales.

\section*{Avances del 2025-12-05}
\subsection*{incorporación del contenido}

En esta fecha se realizó una integración importante al proyecto mediante la incorporación del contenido desde la rama principal del repositorio remoto \texttt{https://github.com/AkaruiKyuketsuki/MusictoSound}. Esta acción representa la consolidación de desarrollos previos y la sincronización del código base con las últimas funcionalidades implementadas.

El propósito principal de esta integración fue unificar el trabajo realizado en distintos entornos de desarrollo para asegurar la coherencia y estabilidad del sistema. La fusión de ramas facilita la detección temprana de conflictos y la consolidación de mejoras, lo que permite mantener la calidad y la integridad del proyecto durante su evolución.

Desde una perspectiva técnica, esta decisión de diseño implica un enfoque de desarrollo colaborativo que prioriza la integración continua, reduciendo riesgos asociados a divergencias prolongadas en el código. Además, la incorporación de los cambios remotos asegura que el sistema se beneficie de las últimas correcciones y optimizaciones, fortaleciendo la base inicial para futuras etapas de desarrollo.

\subsection*{rama principal}

En esta fecha se llevó a cabo la integración del trabajo desarrollado en la rama principal del repositorio del proyecto \textit{MusictoSound}. Esta acción consistió en fusionar los cambios recientes realizados por otros colaboradores con la versión local del proyecto, asegurando así la coherencia y actualización del código base.

El propósito de esta integración fue consolidar los avances distribuidos en diferentes ramas, evitar conflictos futuros y mantener una línea de desarrollo unificada. Esta práctica facilita la colaboración eficiente entre los integrantes del equipo y permite contar con una versión del software que incorpora las últimas funcionalidades y correcciones.

Desde una perspectiva técnica, la fusión implicó la resolución automática o manual de posibles conflictos entre los archivos modificados, garantizando que los módulos del proyecto funcionen correctamente después de la integración. Además, esta acción refuerza la estabilidad del sistema al validar la compatibilidad entre los distintos aportes y permite la continuación del desarrollo sobre una base sólida y sincronizada.

En términos de diseño, se reafirmó la estrategia de uso de control de versiones distribuido para gestionar el desarrollo colaborativo, lo que contribuye a una mayor trazabilidad de las modificaciones y facilita la gestión de cambios en el proyecto.




\section{Avances del 2025-12-08}

\subsection*{Optimización de la conexión con la API en la nueva terminal}
En esta etapa del desarrollo se abordó la mejora y corrección del mecanismo de conexión con la API en el contexto de la actualización a la nueva terminal. La implementación previa presentaba inconsistencias en la comunicación entre el sistema y el servicio externo, lo cual comprometía la estabilidad y eficiencia de la interacción.

Como resultado, se revisaron y ajustaron los controladores y protocolos de conexión para garantizar una correcta autenticación, manejo de sesiones y procesamiento de peticiones. Esta corrección permite una integración más robusta y confiable, minimizando errores y tiempos de espera en la transferencia de datos. Además, se consideraron aspectos de compatibilidad y seguridad para adecuar la conexión a las características de la nueva terminal, asegurando que el sistema mantenga su funcionalidad óptima en el entorno renovado.

La mejora implementada contribuye significativamente a la solidez del sistema, facilitando futuras ampliaciones y manteniendo la calidad en la experiencia de usuario final.

\section{Avances del 2025-12-10}

\subsection*{Integración y fusión de ramas del repositorio principal}
En esta fecha se llevó a cabo la integración de los últimos cambios provenientes de la rama principal del repositorio remoto, mediante una fusión (merge). Este procedimiento es fundamental para mantener la coherencia y actualización del trabajo local respecto al desarrollo centralizado. La combinación de las modificaciones asegura que cualquier mejora, corrección o nueva funcionalidad incorporada en el proyecto original esté reflejada y sincronizada en la versión de trabajo del proyecto de fin de grado.

Desde un punto de vista técnico, esta fusión permite continuar el desarrollo sobre una base consolidada con los últimos avances, reduciendo la posibilidad de conflictos futuros y facilitando una gestión eficiente del código fuente. La operación de merge, aunque en este caso no refleja cambios customizados sino una actualización desde el repositorio principal, es una práctica estándar para mantener la integridad y la calidad del desarrollo en proyectos colaborativos o derivados.

\section{Avances del 2025-12-11}

\subsection*{Integración de la rama principal del repositorio remoto}

En esta etapa del desarrollo se llevó a cabo la integración de la rama principal (main) desde el repositorio remoto alojado en GitHub. Esta acción supuso la sincronización del código local con los últimos cambios presentes en el repositorio central, garantizando la coherencia y actualización del proyecto MusictoSound. La integración permite mantener una base de código homogénea entre los distintos colaboradores y facilita la incorporación de nuevas funcionalidades o la corrección de errores realizados previamente. Realizar esta unión con la rama principal es una práctica fundamental en el flujo de trabajo colaborativo, ya que ayuda a evitar conflictos futuros y asegura que las modificaciones locales se basen en la versión más reciente y estable del proyecto. Esta operación contribuye al mantenimiento del control de versiones y a la continuidad en el desarrollo del trabajo final de grado.

\section{Avances del 2025-12-11}

\subsection*{Integración de la rama principal del repositorio remoto}
En esta fecha se realizó una fusión de la rama principal del repositorio remoto, integrando los últimos cambios realizados en la base de código del proyecto MusictoSound. Este proceso permitió sincronizar el código local con las actualizaciones más recientes, asegurando la coherencia del desarrollo y la incorporación de posibles mejoras, correcciones o nuevas funcionalidades aportadas por otros colaboradores.

La integración de la rama principal es una práctica fundamental en el control de versiones cuando se trabaja en entornos colaborativos o multifacéticos, ya que minimiza conflictos futuros y facilita el seguimiento cronológico y estructurado de las modificaciones. Además, fusionar correctamente garantiza un estado estable y actualizado del proyecto sobre el cual continuar con el desarrollo del Trabajo de Fin de Grado en Ingeniería Informática, evitando desactualizaciones o incompatibilidades con el flujo de trabajo establecido.

\section{Avances del 2025-12-11}

\subsection*{Integración y actualización desde el repositorio principal}
En esta etapa del desarrollo se realizó la integración de los últimos cambios provenientes de la rama principal del repositorio remoto correspondiente al proyecto MusictoSound. Esta acción fue necesaria para mantener el código actualizado con las mejoras y correcciones incorporadas por otros colaboradores o por iteraciones previas del propio proyecto. 

La integración de la rama principal asegura la coherencia y la compatibilidad del código fuente local con la versión más reciente del proyecto, evitando conflictos futuros y facilitando la colaboración en equipo. Además, permite aprovechar las optimizaciones y nuevas funcionalidades añadidas que pueden potenciar el rendimiento y la estabilidad del sistema en desarrollo.

Desde un punto de vista técnico, este procedimiento implica la fusión (merge) controlada de las distintas modificaciones efectuadas en paralelo, lo que requiere una correcta gestión de versiones para preservar la integridad del proyecto. La decisión de realizar esta actualización en una fase concreta del desarrollo responde a la necesidad de sincronización constante en proyectos colaborativos y multifacéticos, garantizando así un avance coherente hacia los objetivos planteados.

\section{Avances del 2025-12-11}

\subsection*{Integración y sincronización con el repositorio principal}
En esta fecha se realizó la integración de los avances locales con la rama principal del repositorio remoto. Este proceso de merge tiene un papel fundamental para mantener la coherencia y actualización continua del proyecto, asegurando que todas las modificaciones desarrolladas tanto local como remotamente estén coordinadas en una única línea de desarrollo. La operación permitió resolver posibles conflictos de versiones y unificar el código base, facilitando la colaboración entre distintos miembros y mejorando la estabilidad general del sistema. Asimismo, este procedimiento optimiza la trazabilidad del proyecto, consolidando las mejoras implementadas y preparando la base para futuras funcionalidades o correcciones.

\section{Avances del 2025-12-13}

\subsection*{Integración de la rama principal del repositorio remoto}
En esta etapa del desarrollo se realizó la integración de los cambios provenientes de la rama principal del repositorio remoto alojado en GitHub. Esta fusión representa un paso importante para mantener el código local actualizado con las últimas modificaciones, garantizando así la coherencia y la sincronización entre las distintas ramas de desarrollo. La incorporación de estos cambios permite evitar conflictos futuros y facilita la colaboración conjunta, asegurando que las funcionalidades desarrolladas se fundamenten sobre una base común y estable. Desde un punto de vista técnico, esta operación implica la resolución automática o manual de posibles discrepancias en el código, la adaptación a nuevas dependencias o configuraciones, así como la consolidación de mejoras implementadas en paralelo por otros colaboradores. Este proceso es esencial para asegurar la integridad y la calidad del proyecto durante su evolución.

\section{Avances del 2025-12-14}

\subsection*{Integración de la rama principal del repositorio remoto}
En esta fecha se realizó la fusión de la rama principal del repositorio remoto hacia la rama local del proyecto MusictoSound. Esta acción tuvo como propósito principal sincronizar los desarrollos más recientes realizados en la rama principal con el entorno de trabajo local, garantizando así que el código fuente disponible estuviera actualizado y reflejara los cambios más recientes implementados por el equipo. Técnicamente, esta integración facilita la consolidación del progreso del proyecto, evita conflictos de versiones y permite que futuras modificaciones se realicen sobre una base estable y común. Además, la actualización contribuye a mantener la coherencia entre los distintos desarrolladores y a agilizar los procesos de desarrollo conjunto. Esta práctica evita la divergencia del código y asegura que las nuevas funcionalidades o correcciones sean accesibles para todos los integrantes del equipo, lo que es fundamental para la correcta evolución del proyecto.
