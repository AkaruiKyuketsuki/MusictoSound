% !TeX root = ../main.tex
% Este capitulo se actualizara automaticamente mediante GitHub Actions
% No escribas nada aqui a mano


% === AUTO-GENERATED ENTRY ===

\section{Commits del 2025-10-01}
Prueba hello world.
\subsection*{Commit 1}
Initial commit.

\section{Commits del 2025-12-04}
\subsection*{Commit 1 del 2025-12-04}
Añadido el registro de cambios internos y la estructura minima de la memoria.
\subsection*{Commit 2 del 2025-12-04}
Creada la estructura tipo latex para docs.
\subsection*{Commit 3 del 2025-12-04}
Creacion inicial de carpetas.

\section{Avances del 2025-12-04}
\subsection*{Funcion de generación de código}
En esta fase del desarrollo, se implementa una correccion intral en la funcion encargada de la generacion de la memoria tecnica del proyecto (\texttt{generate\_memoria}). La modificacion tuvo como objetivo principal mejorar la robustez en el manejo de las interacciones con la API de OpenAI, lo que se traduce en una mayor estabilidad y fiabilidad durante el proceso de generacion automatica de contenidos.

Desde el punto de vista del diseño, se revisaron los mecanismos de gestion de errores y excepciones relacionados con las llamadas a la API, incorporando estrategias que permiten una recuperacion efectiva ante posibles fallos de comunicacion o respuestas inesperadas. Esta decision se tomó para garantizar que el sistema pueda continuar operando sin interrupciones significativas, evitando perdidas de informacion o resultados incompletos.

El impacto de estos ajustes es fundamental para asegurar una experiencia de usuario consistente y mejorar la calidad del output generado, aspectos cruciales para el exito en la presentacion y defensa del Trabajo de Fin de Grado. Ademas, este trabajo facilita futuras ampliaciones o adaptaciones, al contar con una infraestructura de manejo de API mas salida y mantenible.

El impacto de estos ajustes es fundamental para asegurar una experiencia de usuario consistente y mejorar la calidad del output generado, aspectos cruciales para el éxito en la presentación y defensa del Trabajo de Fin de Grado. Además, este trabajo facilita futuras ampliaciones o adaptaciones, al contar con una infraestructura de manejo de API más sólida y mantenible.

\subsection*{integración con la API de OpenAI}

En esta fecha se completó la integración del sistema con la API de OpenAI, lo cual representa un hito crucial en el desarrollo del proyecto. El propósito principal de este avance fue habilitar la comunicación directa entre la plataforma del trabajo y los servicios de inteligencia artificial proporcionados por OpenAI, permitiendo así el uso de modelos avanzados para el procesamiento y generación de texto.

Desde un punto de vista técnico, la decisión de enlazar con la API externa estuvo guiada por la necesidad de aprovechar capacidades de procesamiento natural del lenguaje sin incurrir en el desarrollo de modelos propios, lo que supone un ahorro significativo en recursos y tiempo. La implementación consideró aspectos clave como la gestión eficiente de las solicitudes HTTP, el manejo de respuestas asíncronas y la correcta administración de las claves de acceso para garantizar la seguridad y estabilidad del sistema.

El impacto de esta integración se refleja en la ampliación de las funcionalidades del proyecto, mejorando la calidad y precisión de las tareas automáticas que dependen de la inteligencia artificial, además de establecer una base robusta para futuras extensiones. Se realizaron pruebas exhaustivas para validar la correcta comunicación, respuesta y tolerancia a fallos en el enlace con la API, asegurando así la fiabilidad del sistema en entornos reales.

\section*{Avances del 2025-12-05}
\subsection*{incorporación del contenido}

En esta fecha se realizó una integración importante al proyecto mediante la incorporación del contenido desde la rama principal del repositorio remoto \texttt{https://github.com/AkaruiKyuketsuki/MusictoSound}. Esta acción representa la consolidación de desarrollos previos y la sincronización del código base con las últimas funcionalidades implementadas.

El propósito principal de esta integración fue unificar el trabajo realizado en distintos entornos de desarrollo para asegurar la coherencia y estabilidad del sistema. La fusión de ramas facilita la detección temprana de conflictos y la consolidación de mejoras, lo que permite mantener la calidad y la integridad del proyecto durante su evolución.

Desde una perspectiva técnica, esta decisión de diseño implica un enfoque de desarrollo colaborativo que prioriza la integración continua, reduciendo riesgos asociados a divergencias prolongadas en el código. Además, la incorporación de los cambios remotos asegura que el sistema se beneficie de las últimas correcciones y optimizaciones, fortaleciendo la base inicial para futuras etapas de desarrollo.

\subsection*{rama principal}

En esta fecha se llevó a cabo la integración del trabajo desarrollado en la rama principal del repositorio del proyecto \textit{MusictoSound}. Esta acción consistió en fusionar los cambios recientes realizados por otros colaboradores con la versión local del proyecto, asegurando así la coherencia y actualización del código base.

El propósito de esta integración fue consolidar los avances distribuidos en diferentes ramas, evitar conflictos futuros y mantener una línea de desarrollo unificada. Esta práctica facilita la colaboración eficiente entre los integrantes del equipo y permite contar con una versión del software que incorpora las últimas funcionalidades y correcciones.

Desde una perspectiva técnica, la fusión implicó la resolución automática o manual de posibles conflictos entre los archivos modificados, garantizando que los módulos del proyecto funcionen correctamente después de la integración. Además, esta acción refuerza la estabilidad del sistema al validar la compatibilidad entre los distintos aportes y permite la continuación del desarrollo sobre una base sólida y sincronizada.

En términos de diseño, se reafirmó la estrategia de uso de control de versiones distribuido para gestionar el desarrollo colaborativo, lo que contribuye a una mayor trazabilidad de las modificaciones y facilita la gestión de cambios en el proyecto.




<<<<<<< HEAD
\section{Avances del 2025-12-08}

\subsection*{Optimización de la conexión con la API en la nueva terminal}
En esta etapa del desarrollo se abordó la mejora y corrección del mecanismo de conexión con la API en el contexto de la actualización a la nueva terminal. La implementación previa presentaba inconsistencias en la comunicación entre el sistema y el servicio externo, lo cual comprometía la estabilidad y eficiencia de la interacción.

Como resultado, se revisaron y ajustaron los controladores y protocolos de conexión para garantizar una correcta autenticación, manejo de sesiones y procesamiento de peticiones. Esta corrección permite una integración más robusta y confiable, minimizando errores y tiempos de espera en la transferencia de datos. Además, se consideraron aspectos de compatibilidad y seguridad para adecuar la conexión a las características de la nueva terminal, asegurando que el sistema mantenga su funcionalidad óptima en el entorno renovado.

La mejora implementada contribuye significativamente a la solidez del sistema, facilitando futuras ampliaciones y manteniendo la calidad en la experiencia de usuario final.
=======
Este proceso tiene un impacto positivo en la calidad y estabilidad del software, dado que permite disponer de la versión más actualizada, reduciendo la posibilidad de conflictos futuros y asegurando que las funcionalidades desarrolladas por distintos colaboradores se integren de manera eficiente. Además, plantea un punto de partida sólido para las siguientes fases del desarrollo y para la implementación de nuevas características.

\section{Avances del 2025-12-08}

\subsection*{Commit 2c0ac3a}
En este commit se llevó a cabo la integración de la rama principal del repositorio remoto correspondiente al proyecto \textit{MusictoSound}. Esta operación representa un paso fundamental en la consolidación del desarrollo, asegurando que los cambios realizados localmente estén alineados con el estado más reciente del proyecto compartido.

Desde una perspectiva técnica, la fusión implica la resolución de posibles conflictos de código y la incorporación de nuevas funcionalidades o correcciones documentadas en la rama remota. Este proceso es crucial para mantener la coherencia del código y evitar divergencias que puedan afectar la estabilidad o escalabilidad del software.

La decisión de realizar esta integración en una etapa avanzada del desarrollo favorece la reducción de errores derivados de desincronizaciones y permite un flujo de trabajo colaborativo más eficiente. Además, refleja una buena práctica de gestión de versiones dentro del ciclo de vida del proyecto, facilitando futuras iteraciones y pruebas del sistema en un entorno actualizado y consolidado.
>>>>>>> a4d71c39d687725fbaade8af2366813ab72874a1
