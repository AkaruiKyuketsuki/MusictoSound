% !TeX root = ../main.tex
% Este capitulo se actualizara automaticamente mediante GitHub Actions
% No escribas nada aqui a mano


% === AUTO-GENERATED ENTRY ===

\section{Commits del 2025-10-01}
Prueba hello world.
\subsection*{Commit 1}
Initial commit.

\section{Commits del 2025-12-04}
\subsection*{Commit 1}
Añadido el registro de cambios internos y la estructura minima de la memoria.
\subsection*{Commit 2}
Creada la estructura tipo latex para docs.
\subsection*{Commit 3}
Creacion inicial de carpetas.

\section{Avances del 2025-12-04}
\subsection*{Funcion de generación de código}
En esta fase del desarrollo, se implementa una correccion intral en la funcion encargada de la generacion de la memoria tecnica del proyecto (\texttt{generate\_memoria}). La modificacion tuvo como objetivo principal mejorar la robustez en el manejo de las interacciones con la API de OpenAI, lo que se traduce en una mayor estabilidad y fiabilidad durante el proceso de generacion automatica de contenidos.

Desde el punto de vista del diseño, se revisaron los mecanismos de gestion de errores y excepciones relacionados con las llamadas a la API, incorporando estrategias que permiten una recuperacion efectiva ante posibles fallos de comunicacion o respuestas inesperadas. Esta decision se tomó para garantizar que el sistema pueda continuar operando sin interrupciones significativas, evitando perdidas de informacion o resultados incompletos.

El impacto de estos ajustes es fundamental para asegurar una experiencia de usuario consistente y mejorar la calidad del output generado, aspectos cruciales para el exito en la presentacion y defensa del Trabajo de Fin de Grado. Ademas, este trabajo facilita futuras ampliaciones o adaptaciones, al contar con una infraestructura de manejo de API mas salida y mantenible.

El impacto de estos ajustes es fundamental para asegurar una experiencia de usuario consistente y mejorar la calidad del output generado, aspectos cruciales para el éxito en la presentación y defensa del Trabajo de Fin de Grado. Además, este trabajo facilita futuras ampliaciones o adaptaciones, al contar con una infraestructura de manejo de API más sólida y mantenible.

\subsection*{integración con la API de OpenAI}

En esta fecha se completó la integración del sistema con la API de OpenAI, lo cual representa un hito crucial en el desarrollo del proyecto. El propósito principal de este avance fue habilitar la comunicación directa entre la plataforma del trabajo y los servicios de inteligencia artificial proporcionados por OpenAI, permitiendo así el uso de modelos avanzados para el procesamiento y generación de texto.

Desde un punto de vista técnico, la decisión de enlazar con la API externa estuvo guiada por la necesidad de aprovechar capacidades de procesamiento natural del lenguaje sin incurrir en el desarrollo de modelos propios, lo que supone un ahorro significativo en recursos y tiempo. La implementación consideró aspectos clave como la gestión eficiente de las solicitudes HTTP, el manejo de respuestas asíncronas y la correcta administración de las claves de acceso para garantizar la seguridad y estabilidad del sistema.

El impacto de esta integración se refleja en la ampliación de las funcionalidades del proyecto, mejorando la calidad y precisión de las tareas automáticas que dependen de la inteligencia artificial, además de establecer una base robusta para futuras extensiones. Se realizaron pruebas exhaustivas para validar la correcta comunicación, respuesta y tolerancia a fallos en el enlace con la API, asegurando así la fiabilidad del sistema en entornos reales.

\section*{Avances del 2025-12-05}
\subsection*{incorporación del contenido}

En esta fecha se realizó una integración importante al proyecto mediante la incorporación del contenido desde la rama principal del repositorio remoto \texttt{https://github.com/AkaruiKyuketsuki/MusictoSound}. Esta acción representa la consolidación de desarrollos previos y la sincronización del código base con las últimas funcionalidades implementadas.

El propósito principal de esta integración fue unificar el trabajo realizado en distintos entornos de desarrollo para asegurar la coherencia y estabilidad del sistema. La fusión de ramas facilita la detección temprana de conflictos y la consolidación de mejoras, lo que permite mantener la calidad y la integridad del proyecto durante su evolución.

Desde una perspectiva técnica, esta decisión de diseño implica un enfoque de desarrollo colaborativo que prioriza la integración continua, reduciendo riesgos asociados a divergencias prolongadas en el código. Además, la incorporación de los cambios remotos asegura que el sistema se beneficie de las últimas correcciones y optimizaciones, fortaleciendo la base inicial para futuras etapas de desarrollo.

\subsection*{rama principal}

En esta fecha se llevó a cabo la integración del trabajo desarrollado en la rama principal del repositorio del proyecto \textit{MusictoSound}. Esta acción consistió en fusionar los cambios recientes realizados por otros colaboradores con la versión local del proyecto, asegurando así la coherencia y actualización del código base.

El propósito de esta integración fue consolidar los avances distribuidos en diferentes ramas, evitar conflictos futuros y mantener una línea de desarrollo unificada. Esta práctica facilita la colaboración eficiente entre los integrantes del equipo y permite contar con una versión del software que incorpora las últimas funcionalidades y correcciones.

Desde una perspectiva técnica, la fusión implicó la resolución automática o manual de posibles conflictos entre los archivos modificados, garantizando que los módulos del proyecto funcionen correctamente después de la integración. Además, esta acción refuerza la estabilidad del sistema al validar la compatibilidad entre los distintos aportes y permite la continuación del desarrollo sobre una base sólida y sincronizada.

En términos de diseño, se reafirmó la estrategia de uso de control de versiones distribuido para gestionar el desarrollo colaborativo, lo que contribuye a una mayor trazabilidad de las modificaciones y facilita la gestión de cambios en el proyecto.




\section{Avances del 2025-12-08}

\subsection*{Optimización de la conexión con la API en la nueva terminal}
En esta etapa del desarrollo se abordó la mejora y corrección del mecanismo de conexión con la API en el contexto de la actualización a la nueva terminal. La implementación previa presentaba inconsistencias en la comunicación entre el sistema y el servicio externo, lo cual comprometía la estabilidad y eficiencia de la interacción.

Como resultado, se revisaron y ajustaron los controladores y protocolos de conexión para garantizar una correcta autenticación, manejo de sesiones y procesamiento de peticiones. Esta corrección permite una integración más robusta y confiable, minimizando errores y tiempos de espera en la transferencia de datos. Además, se consideraron aspectos de compatibilidad y seguridad para adecuar la conexión a las características de la nueva terminal, asegurando que el sistema mantenga su funcionalidad óptima en el entorno renovado.

La mejora implementada contribuye significativamente a la solidez del sistema, facilitando futuras ampliaciones y manteniendo la calidad en la experiencia de usuario final.

\section{Avances del 2025-10-01}

\subsection*{Inicialización del repositorio y prueba básica}
En esta fecha se realizaron los primeros pasos para la creación del proyecto. Se efectuó el commit inicial que estableció la base del repositorio, incluyendo su estructura básica y los archivos fundamentales para comenzar el desarrollo. Además, se implementó una prueba inicial del tipo "hello world" para verificar el correcto funcionamiento del entorno de desarrollo y asegurar que la integración básica del código fuese efectiva. Estas acciones sentaron las bases para futuros desarrollos, estableciendo un entorno estable y controlado desde el cual continuar el trabajo.

\section{Avances del 2025-12-04}

\subsection*{Creación de la estructura inicial del proyecto y documentación en LaTeX}
Durante esta jornada se creó la estructura de carpetas necesaria para el buen orden del proyecto, lo que facilitó el manejo de los diferentes componentes y recursos asociados. Se diseñó una estructura base en LaTeX para la documentación, estableciendo un formato uniforme que asegurará la coherencia y profesionalidad del trabajo escrito. Además, se añadió el registro de cambios internos que permite mantener un seguimiento detallado de las modificaciones realizadas a lo largo del desarrollo.

\subsection*{Desarrollo y mejora del script generador automático de memoria}
Se desarrolló un script encargado de generar automáticamente la memoria del proyecto, facilitando la actualización constante y sistemática de la documentación técnica. Se integró la conexión con la API de OpenAI para potenciar esta generación automática, adoptando un manejo robusto y actualizado del SDK de OpenAI. Esta integración aseguró una síntesis adecuada y automatizada de los avances del proyecto, mejorando la eficiencia del proceso de documentación.

\subsection*{Configuración y automatización de integración continua}
Se estableció un flujo de trabajo para la integración continua que permite la generación automática de la memoria en formato PDF. Esto garantiza que la documentación esté siempre actualizada y disponible en un formato estándar y fácil de distribuir. Además, se realizaron ajustes para corregir errores tipográficos y de acentuación en el texto, asegurando la calidad y legibilidad del documento final.

\subsection*{Ajustes en la conexión con la API y mejoras de la interfaz}
Se corrigieron problemas en la conexión con la API al adaptarla a una nueva terminal, asegurando la estabilidad y fiabilidad de la comunicación entre los componentes del proyecto. También se descargaron e incorporaron extensiones de LaTeX para mejorar la visualización y generación de los documentos, además de cambiar y optimizar la estructura interna de LaTeX para un mejor manejo y presentación del contenido.

\section{Avances del 2025-12-08}

\subsection*{Mejoras en el estilo y formato de los capítulos de la memoria}
Se finalizaron y aplicaron ajustes significativos en el estilo visual y el formato textual de los capítulos de la memoria. Este trabajo incluyó pruebas para la correcta visualización de títulos y secciones internas de los capítulos, facilitando la navegación y lectura del documento. Se implementó funcionalidad para redireccionamientos en el índice sin la aparición de elementos visuales no deseados, lo que mejora la experiencia de uso y la presentación general del documento.

\subsection*{Corrección de errores y refactorización del código}
Se llevó a cabo una exhaustiva corrección de errores detectados en etapas anteriores, incluyendo aspectos relacionados con tipologías y acentuación, asegurando la calidad técnica y ortográfica de la memoria. Además, se perfeccionó la conexión con la API en la nueva terminal, garantizando una comunicación robusta y eficiente bajo el nuevo entorno.

\subsection*{Integración continua y actualización automatizada de la documentación}
Diversos commits correspondieron a la configuración y mejora de procesos automáticos para la actualización continua de la memoria, fortaleciendo el flujo de trabajo y asegurando que la documentación esté siempre sincronizada con los avances más recientes del proyecto.

\section{Avances del 2025-12-10}

\subsection*{Implementación del modo manual de ejecución para Audiveris}
Se añadió la funcionalidad para abrir Audiveris mediante la ejecución del programa con la opción 2, que corresponde al modo manual. Esta mejora permite al usuario controlar de forma directa y personalizada el proceso de reconocimiento óptico de música, lo que puede ser esencial para tareas que requieren una intervención más precisa o ajustes específicos durante la ejecución.

\subsection*{Finalización de la tarea inicial de creación del proyecto}
Se completó la tarea 0, que consistió en la creación inicial del proyecto estableciendo su base funcional y organizativa. Esta fase fue crucial para definir las líneas principales de desarrollo y las herramientas a emplear, asegurándose de tener un entorno adecuado para continuar con las siguientes etapas del proyecto.

\subsection*{Integración y resolución de conflictos mediante merges múltiples}
Se realizaron varias integraciones de ramas utilizando merges para unificar las funcionalidades y correcciones implementadas en diferentes ramas de desarrollo. Estas operaciones permitieron consolidar el progreso realizado, incorporando correcciones y mejoras previas que optimizan la estabilidad y funcionalidad general del proyecto.
