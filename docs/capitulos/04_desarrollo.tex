% !TeX root = ../main.tex
% Este capitulo se actualizara automaticamente mediante GitHub Actions
% No escribas nada aqui a mano


% === AUTO-GENERATED ENTRY ===
\section{Avances del 2025-10-01}

\subsection*{Inicialización y pruebas básicas del proyecto}
En esta fecha se establecieron las bases iniciales del proyecto con el primer commit identificado como \textit{Initial commit}, donde se creó la estructura fundamental del repositorio. Se realizaron pruebas preliminares básicas, incluyendo la implementación de un programa sencillo tipo "Hello World". Estas acciones fueron esenciales para garantizar el correcto funcionamiento del entorno de desarrollo y la gestión de versiones, constituyendo el punto de partida del desarrollo posterior.


\section{Avances del 2025-12-04}

\subsection*{Creación inicial de carpetas y estructura de documentación en LaTeX}
Se sentaron las bases organizativas del proyecto mediante la creación de las carpetas iniciales necesarias para el correcto almacenamiento de código, documentación y otros recursos. Además, se diseñó una estructura básica de documentos con formato LaTeX, destinada a la elaboración de la memoria académica, buscando un formato profesional y acorde a los estándares académicos.

\subsection*{Definición de la memoria mínima y registro de cambios}
Se incorporó una estructura mínima para la memoria del proyecto, incluyendo un registro de cambios interno que permite la trazabilidad de las modificaciones y avances. Este registro será clave para la gestión documental y para facilitar la revisión por parte de tutores y evaluadores.

\subsection*{Automatización de la generación de la memoria académica}
Se desarrolló un script automatizado para la generación de la memoria, que permite crear documentos en PDF con contenido actualizado de forma automática. Esta funcionalidad optimiza el proceso de documentación, reduciendo errores manuales y proporcionando una referencia puntual de los avances realizados.

\subsection*{Integración y configuración de la API de OpenAI}
Se completó el enlace con la API de OpenAI, lo que habilita capacidades de generación automática de textos académicos y asistencia en la elaboración de la memoria. Se realizaron correcciones para asegurar un manejo robusto de las respuestas y se actualizó el SDK para un acceso óptimo a las funciones de chat completions, garantizando así una interacción fluida con la inteligencia artificial.

\subsection*{Ajustes en la estructura y visualización del documento LaTeX}
Se realizaron cambios significativos en la estructura interna del documento LaTeX, mejorando el formato y el estilo de los capítulos y secciones. El formato del texto fue revisado para cumplir con criterios de legibilidad y presentación académica, superando errores previos de tipografía y acentuación. También se añadió un visor para LaTeX, facilitando la revisión inmediata de los cambios realizados en la documentación.

\subsection*{Mejoras en la conexión con la API y pruebas de generación automática}
Se corrigieron problemas en la conexión con la API dentro de la nueva terminal utilizada, asegurando una comunicación estable y eficiente. Asimismo, se realizaron pruebas para validar la generación automática de capítulos con títulos adecuados y la correcta creación de secciones internas en la memoria, comprobando la funcionalidad completa y acorde con las expectativas del proyecto.


\section{Avances del 2025-12-05}

\subsection*{Cambio y consolidación del título y generación de la memoria en PDF}
Se definió y ajustó el título del proyecto para reflejar de manera precisa el contenido y propósito del trabajo. Además, se logró la generación efectiva de la memoria en formato PDF, lo que demuestra la consolidación de la documentación automatizada y su compatibilidad con los estándares académicos requeridos.

\subsection*{Descarga e integración de extensiones LaTeX}
Se incorporaron extensiones adicionales para LaTeX, mejorando así las capacidades de formato, visualización y edición del documento. Estos complementos facilitarán la creación de una memoria más pulida y profesional.

\subsection*{Mejoras en la estructura de LaTeX y visualización integrada}
Se realizaron modificaciones en la estructura de LaTeX para optimizar la organización de los contenidos, así como el soporte para visualización directa, lo cual agiliza el proceso iterativo de edición y revisión del documento.

\subsection*{Múltiples actualizaciones automatizadas de la memoria}
Durante este día se realizaron varias actualizaciones automáticas consecutivas de la memoria, evidenciando la puesta en marcha efectiva del sistema de integración continua configurado para mantener el contenido de la documentación siempre actualizado con los últimos avances del proyecto.


\section{Avances del 2025-12-08}

\subsection*{Refinamientos en el estilo y formato de la memoria}
Se concluyó el diseño del estilo visual de los capítulos y el formato general de texto, mejorando la coherencia estética y facilitando la lectura. Se realizaron pruebas sucesivas para ajustar la organización de capítulos y secciones, así como para perfeccionar la funcionalidad de la navegación en el índice, eliminando elementos distractores como recuadros rojos en los enlaces.

\subsection*{Corrección de errores tipográficos y de conexionado}
Se corrigieron errores de acentuación y tipologías detectados previamente, mejorando la calidad del documento. Además, se solventaron problemas en la conexión con la API de OpenAI en la nueva terminal, asegurando un acceso más robusto y estable a dicha herramienta.

\subsection*{Consolidación del repositorio y gestión de versiones}
Se continuó con la integración y fusión de ramas de desarrollo, manteniendo la coherencia en el repositorio y restaurando cambios previos cuando fue necesario. Esta práctica aseguró la conservación del progreso y la estabilidad del estado del proyecto.

\subsection*{Pruebas y mejoras continuas en la generación de contenido}
Diversas pruebas fueron ejecutadas para validar la generación automática de capítulos y memorias con títulos adecuados, secuenciación correcta y formato acorde. Este proceso de iteración permitió detectar y resolver inconsistencias, redefiniendo la estructura interna para una mejor experiencia de edición y consulta.


\section{Avances del 2025-12-10}

\subsection*{Finalización de la tarea inicial de creación del proyecto}
Se completó con éxito la primera tarea relacionada con la creación y establecimiento del proyecto, consolidando las bases necesarias para el desarrollo posterior.

\subsection*{Ejecución de Audiveris en modo manual}
Se configuró la apertura y ejecución del software Audiveris en modo manual mediante la ejecución del \textit{main} con la opción específica para tal modo. Esto permite una interacción directa con el proceso de transcripción de partituras, habilitando la edición y control manual del reconocimiento óptico.

\subsection*{Reestructuración y fusiones para sincronización con rama principal}
Se efectuaron diversas fusiones con la rama principal para mantener la sincronización del proyecto y asegurar la integración de las últimas mejoras y correcciones.

\subsection*{Actualizaciones automáticas y gestión continua}
Se registraron múltiples actualizaciones automáticas que reflejan una mejora continua en la generación y gestión de la memoria, mostrando la efectiva implementación de pipelines de integración continua.


\section{Avances del 2025-12-11}

\subsection*{Implementación del patrón Modelo-Vista-Controlador (MVC) en la interfaz}
Se incorporó el patrón arquitectónico Modelo-Vista-Controlador en el diseño de la interfaz gráfica, asegurando una separación clara entre la lógica de negocio, la presentación y el control de eventos. Esta decisión técnica facilita el mantenimiento, escalabilidad y testeo del software, además de mejorar la modularidad del desarrollo.

\subsection*{Desarrollo de modos automático y manual para transcripción de PDF a XML}
Se implementaron dos modos de operación para la transcripción de documentos PDF a formato XML: un modo automático y otro manual. Ambas modalidades utilizan llamadas a Audiveris a través de procesos secundario (subprocess) ejecutados desde la consola, permitiendo flexibilidad en el procesamiento según las necesidades del usuario.

\subsection*{Restauración y consolidación de cambios previos}
Se recuperaron modificaciones anteriores que pudieron haberse visto comprometidas durante fusiones o reestructuraciones, garantizando la estabilidad e integridad del código base.

\subsection*{Creación básica de la interfaz gráfica con Tkinter}
Se desarrolló una interfaz gráfica básica utilizando la biblioteca Tkinter de Python, orientada a la funcionalidad de transcripción. Esta primera versión permite la interacción con botones y comandos, estableciendo las bases para futuras ampliaciones y mejoras en la experiencia de usuario.

\subsection*{Conexión entre la interfaz y elementos interactivos}
Se completó la integración funcional entre la interfaz y los botones, asegurando que las acciones del usuario sean correctamente capturadas y procesadas. Esto implica la configuración adecuada de controladores de eventos y la comunicación fluida entre los componentes visuales y la lógica de aplicación.

\subsection*{Fusiones finales y actualización del repositorio}
Se consolidaron los cambios mediante fusiones con la rama principal, actualizando el repositorio remoto para reflejar el estado más reciente y estable del proyecto, preparándolo para las siguientes etapas de desarrollo.


\section{Avances del 2025-12-13}

\subsection*{Integración final de la estructura e interfaz del proyecto}
Se fusionaron las ramas principales del repositorio, garantizando que los últimos desarrollos en estructura y funcionalidades estén integrados en la versión definitiva del código.

\subsection*{Consolidación de la interfaz gráfica y operaciones de botones}
Se afianzó la conexión entre la interfaz y los botones, confirmando que los elementos visuales respondan adecuadamente a las entradas del usuario. Esto mejora la usabilidad y aporta una experiencia más intuitiva para la interacción con el sistema.

<<<<<<< HEAD
\subsection*{Integración y sincronización con el repositorio principal}
En esta fecha se realizó la integración de los avances locales con la rama principal del repositorio remoto. Este proceso de merge tiene un papel fundamental para mantener la coherencia y actualización continua del proyecto, asegurando que todas las modificaciones desarrolladas tanto local como remotamente estén coordinadas en una única línea de desarrollo. La operación permitió resolver posibles conflictos de versiones y unificar el código base, facilitando la colaboración entre distintos miembros y mejorando la estabilidad general del sistema. Asimismo, este procedimiento optimiza la trazabilidad del proyecto, consolidando las mejoras implementadas y preparando la base para futuras funcionalidades o correcciones.

\section{Avances del 2025-12-13}

\subsection*{Integración de la rama principal del repositorio remoto}
En esta etapa del desarrollo se realizó la integración de los cambios provenientes de la rama principal del repositorio remoto alojado en GitHub. Esta fusión representa un paso importante para mantener el código local actualizado con las últimas modificaciones, garantizando así la coherencia y la sincronización entre las distintas ramas de desarrollo. La incorporación de estos cambios permite evitar conflictos futuros y facilita la colaboración conjunta, asegurando que las funcionalidades desarrolladas se fundamenten sobre una base común y estable. Desde un punto de vista técnico, esta operación implica la resolución automática o manual de posibles discrepancias en el código, la adaptación a nuevas dependencias o configuraciones, así como la consolidación de mejoras implementadas en paralelo por otros colaboradores. Este proceso es esencial para asegurar la integridad y la calidad del proyecto durante su evolución.

\section{Avances del 2025-12-14}

\subsection*{Integración de la rama principal del repositorio remoto}
En esta fecha se realizó la fusión de la rama principal del repositorio remoto hacia la rama local del proyecto MusictoSound. Esta acción tuvo como propósito principal sincronizar los desarrollos más recientes realizados en la rama principal con el entorno de trabajo local, garantizando así que el código fuente disponible estuviera actualizado y reflejara los cambios más recientes implementados por el equipo. Técnicamente, esta integración facilita la consolidación del progreso del proyecto, evita conflictos de versiones y permite que futuras modificaciones se realicen sobre una base estable y común. Además, la actualización contribuye a mantener la coherencia entre los distintos desarrolladores y a agilizar los procesos de desarrollo conjunto. Esta práctica evita la divergencia del código y asegura que las nuevas funcionalidades o correcciones sean accesibles para todos los integrantes del equipo, lo que es fundamental para la correcta evolución del proyecto.
