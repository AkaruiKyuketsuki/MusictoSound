% !TeX root = ../main.tex
% Este capítulo se actualizará automáticamente mediante GitHub Actions
% No escribas nada aquí a mano


% === AUTO-GENERATED ENTRY ===
\section*{Commit del 2025-12-04}
Aã±adido el registro de cambios internos y la estructura minima de la memoria.

\section*{Commit del 2025-12-04}
Creada la estructura tipo latex para docs.

\section*{Commit del 2025-12-04}
Creacion inicial de carpetas.

\section*{Commit del 2025-10-01}
Prueba hello world.

\section*{Commit del 2025-10-01}
Initial commit.


\section*{Avances del 2025-12-04}

En esta fase del desarrollo, se implementó una corrección integral en la función encargada de la generación de la memoria técnica del proyecto (\texttt{generate\_memoria}). La modificación tuvo como objetivo principal mejorar la robustez en el manejo de las interacciones con la API de OpenAI, lo que se traduce en una mayor estabilidad y fiabilidad durante el proceso de generación automática de contenidos.

Desde el punto de vista del diseño, se revisaron los mecanismos de gestión de errores y excepciones relacionados con las llamadas a la API, incorporando estrategias que permiten una recuperación efectiva ante posibles fallos de comunicación o respuestas inesperadas. Esta decisión se tomó para garantizar que el sistema pueda continuar operando sin interrupciones significativas, evitando pérdidas de información o resultados incompletos.

El impacto de estos ajustes es fundamental para asegurar una experiencia de usuario consistente y mejorar la calidad del output generado, aspectos cruciales para el éxito en la presentación y defensa del Trabajo de Fin de Grado. Además, este trabajo facilita futuras ampliaciones o adaptaciones, al contar con una infraestructura de manejo de API más sólida y mantenible.

