% !TeX root = ../main.tex
% Este capitulo se actualizara automaticamente mediante GitHub Actions
% No escribas nada aqui a mano


% === AUTO-GENERATED ENTRY ===
\section{Avances del 2025-10-01}

\subsection*{Inicialización y pruebas básicas del proyecto}
En esta fecha se establecieron las bases iniciales del proyecto con el primer commit identificado como \textit{Initial commit}, donde se creó la estructura fundamental del repositorio. Se realizaron pruebas preliminares básicas, incluyendo la implementación de un programa sencillo tipo "Hello World". Estas acciones fueron esenciales para garantizar el correcto funcionamiento del entorno de desarrollo y la gestión de versiones, constituyendo el punto de partida del desarrollo posterior.


\section{Avances del 2025-12-04}

\subsection*{Creación inicial de carpetas y estructura de documentación en LaTeX}
Se sentaron las bases organizativas del proyecto mediante la creación de las carpetas iniciales necesarias para el correcto almacenamiento de código, documentación y otros recursos. Además, se diseñó una estructura básica de documentos con formato LaTeX, destinada a la elaboración de la memoria académica, buscando un formato profesional y acorde a los estándares académicos.

\subsection*{Definición de la memoria mínima y registro de cambios}
Se incorporó una estructura mínima para la memoria del proyecto, incluyendo un registro de cambios interno que permite la trazabilidad de las modificaciones y avances. Este registro será clave para la gestión documental y para facilitar la revisión por parte de tutores y evaluadores.

\subsection*{Automatización de la generación de la memoria académica}
Se desarrolló un script automatizado para la generación de la memoria, que permite crear documentos en PDF con contenido actualizado de forma automática. Esta funcionalidad optimiza el proceso de documentación, reduciendo errores manuales y proporcionando una referencia puntual de los avances realizados.

\subsection*{Integración y configuración de la API de OpenAI}
Se completó el enlace con la API de OpenAI, lo que habilita capacidades de generación automática de textos académicos y asistencia en la elaboración de la memoria. Se realizaron correcciones para asegurar un manejo robusto de las respuestas y se actualizó el SDK para un acceso óptimo a las funciones de chat completions, garantizando así una interacción fluida con la inteligencia artificial.

\subsection*{Ajustes en la estructura y visualización del documento LaTeX}
Se realizaron cambios significativos en la estructura interna del documento LaTeX, mejorando el formato y el estilo de los capítulos y secciones. El formato del texto fue revisado para cumplir con criterios de legibilidad y presentación académica, superando errores previos de tipografía y acentuación. También se añadió un visor para LaTeX, facilitando la revisión inmediata de los cambios realizados en la documentación.

\subsection*{Mejoras en la conexión con la API y pruebas de generación automática}
Se corrigieron problemas en la conexión con la API dentro de la nueva terminal utilizada, asegurando una comunicación estable y eficiente. Asimismo, se realizaron pruebas para validar la generación automática de capítulos con títulos adecuados y la correcta creación de secciones internas en la memoria, comprobando la funcionalidad completa y acorde con las expectativas del proyecto.


\section{Avances del 2025-12-05}

\subsection*{Cambio y consolidación del título y generación de la memoria en PDF}
Se definió y ajustó el título del proyecto para reflejar de manera precisa el contenido y propósito del trabajo. Además, se logró la generación efectiva de la memoria en formato PDF, lo que demuestra la consolidación de la documentación automatizada y su compatibilidad con los estándares académicos requeridos.

\subsection*{Descarga e integración de extensiones LaTeX}
Se incorporaron extensiones adicionales para LaTeX, mejorando así las capacidades de formato, visualización y edición del documento. Estos complementos facilitarán la creación de una memoria más pulida y profesional.

\subsection*{Mejoras en la estructura de LaTeX y visualización integrada}
Se realizaron modificaciones en la estructura de LaTeX para optimizar la organización de los contenidos, así como el soporte para visualización directa, lo cual agiliza el proceso iterativo de edición y revisión del documento.

\subsection*{Múltiples actualizaciones automatizadas de la memoria}
Durante este día se realizaron varias actualizaciones automáticas consecutivas de la memoria, evidenciando la puesta en marcha efectiva del sistema de integración continua configurado para mantener el contenido de la documentación siempre actualizado con los últimos avances del proyecto.


\section{Avances del 2025-12-08}

\subsection*{Refinamientos en el estilo y formato de la memoria}
Se concluyó el diseño del estilo visual de los capítulos y el formato general de texto, mejorando la coherencia estética y facilitando la lectura. Se realizaron pruebas sucesivas para ajustar la organización de capítulos y secciones, así como para perfeccionar la funcionalidad de la navegación en el índice, eliminando elementos distractores como recuadros rojos en los enlaces.

\subsection*{Corrección de errores tipográficos y de conexionado}
Se corrigieron errores de acentuación y tipologías detectados previamente, mejorando la calidad del documento. Además, se solventaron problemas en la conexión con la API de OpenAI en la nueva terminal, asegurando un acceso más robusto y estable a dicha herramienta.

\subsection*{Consolidación del repositorio y gestión de versiones}
Se continuó con la integración y fusión de ramas de desarrollo, manteniendo la coherencia en el repositorio y restaurando cambios previos cuando fue necesario. Esta práctica aseguró la conservación del progreso y la estabilidad del estado del proyecto.

\subsection*{Pruebas y mejoras continuas en la generación de contenido}
Diversas pruebas fueron ejecutadas para validar la generación automática de capítulos y memorias con títulos adecuados, secuenciación correcta y formato acorde. Este proceso de iteración permitió detectar y resolver inconsistencias, redefiniendo la estructura interna para una mejor experiencia de edición y consulta.


\section{Avances del 2025-12-10}

\subsection*{Finalización de la tarea inicial de creación del proyecto}
Se completó con éxito la primera tarea relacionada con la creación y establecimiento del proyecto, consolidando las bases necesarias para el desarrollo posterior.

\subsection*{Ejecución de Audiveris en modo manual}
Se configuró la apertura y ejecución del software Audiveris en modo manual mediante la ejecución del \textit{main} con la opción específica para tal modo. Esto permite una interacción directa con el proceso de transcripción de partituras, habilitando la edición y control manual del reconocimiento óptico.

\subsection*{Reestructuración y fusiones para sincronización con rama principal}
Se efectuaron diversas fusiones con la rama principal para mantener la sincronización del proyecto y asegurar la integración de las últimas mejoras y correcciones.

\subsection*{Actualizaciones automáticas y gestión continua}
Se registraron múltiples actualizaciones automáticas que reflejan una mejora continua en la generación y gestión de la memoria, mostrando la efectiva implementación de pipelines de integración continua.


\section{Avances del 2025-12-11}

\subsection*{Implementación del patrón Modelo-Vista-Controlador (MVC) en la interfaz}
Se incorporó el patrón arquitectónico Modelo-Vista-Controlador en el diseño de la interfaz gráfica, asegurando una separación clara entre la lógica de negocio, la presentación y el control de eventos. Esta decisión técnica facilita el mantenimiento, escalabilidad y testeo del software, además de mejorar la modularidad del desarrollo.

\subsection*{Desarrollo de modos automático y manual para transcripción de PDF a XML}
Se implementaron dos modos de operación para la transcripción de documentos PDF a formato XML: un modo automático y otro manual. Ambas modalidades utilizan llamadas a Audiveris a través de procesos secundario (subprocess) ejecutados desde la consola, permitiendo flexibilidad en el procesamiento según las necesidades del usuario.

\subsection*{Restauración y consolidación de cambios previos}
Se recuperaron modificaciones anteriores que pudieron haberse visto comprometidas durante fusiones o reestructuraciones, garantizando la estabilidad e integridad del código base.

\subsection*{Creación básica de la interfaz gráfica con Tkinter}
Se desarrolló una interfaz gráfica básica utilizando la biblioteca Tkinter de Python, orientada a la funcionalidad de transcripción. Esta primera versión permite la interacción con botones y comandos, estableciendo las bases para futuras ampliaciones y mejoras en la experiencia de usuario.

\subsection*{Conexión entre la interfaz y elementos interactivos}
Se completó la integración funcional entre la interfaz y los botones, asegurando que las acciones del usuario sean correctamente capturadas y procesadas. Esto implica la configuración adecuada de controladores de eventos y la comunicación fluida entre los componentes visuales y la lógica de aplicación.

\subsection*{Fusiones finales y actualización del repositorio}
Se consolidaron los cambios mediante fusiones con la rama principal, actualizando el repositorio remoto para reflejar el estado más reciente y estable del proyecto, preparándolo para las siguientes etapas de desarrollo.


\section{Avances del 2025-12-13}

\subsection*{Integración final de la estructura e interfaz del proyecto}
Se fusionaron las ramas principales del repositorio, garantizando que los últimos desarrollos en estructura y funcionalidades estén integrados en la versión definitiva del código.

\subsection*{Consolidación de la interfaz gráfica y operaciones de botones}
Se afianzó la conexión entre la interfaz y los botones, confirmando que los elementos visuales respondan adecuadamente a las entradas del usuario. Esto mejora la usabilidad y aporta una experiencia más intuitiva para la interacción con el sistema.

\section{Avances del 2025-12-14}

\subsection*{Integración de la rama principal del repositorio remoto}
En esta fecha se realizó la fusión de la rama principal del repositorio remoto hacia la rama local del proyecto MusictoSound. Esta acción tuvo como propósito principal sincronizar los desarrollos más recientes realizados en la rama principal con el entorno de trabajo local, garantizando así que el código fuente disponible estuviera actualizado y reflejara los cambios más recientes implementados por el equipo. Técnicamente, esta integración facilita la consolidación del progreso del proyecto, evita conflictos de versiones y permite que futuras modificaciones se realicen sobre una base estable y común. Además, la actualización contribuye a mantener la coherencia entre los distintos desarrolladores y a agilizar los procesos de desarrollo conjunto. Esta práctica evita la divergencia del código y asegura que las nuevas funcionalidades o correcciones sean accesibles para todos los integrantes del equipo, lo que es fundamental para la correcta evolución del proyecto.

\subsection*{Optimización y corrección de gestión de memoria}
En esta etapa del desarrollo se resolvieron diversos problemas relacionados con la gestión de la memoria dentro del sistema implementado. La correcta administración de la memoria es crucial para garantizar la estabilidad y eficiencia de la aplicación, especialmente en proyectos de ingeniería informática donde el procesamiento intensivo puede provocar fugas o fragmentación.

El análisis de estos problemas reveló inconsistencias en la asignación y liberación de recursos dinámicos, lo que podía llevar a pérdidas progresivas de memoria y, en consecuencia, a una degradación del rendimiento o fallos inesperados en la ejecución. Se aplicaron técnicas de depuración y profiling para identificar los puntos conflictivos y se rediseñaron los manejadores de memoria para asegurar un control riguroso y seguro, evitando el acceso a zonas inválidas y asegurando la liberación correcta de los bloques utilizados.

Este avance mejora significativamente la robustez del sistema, disminuyendo la probabilidad de errores en tiempo de ejecución y facilitando el mantenimiento futuro. Además, un manejo de memoria eficiente es fundamental para el escalado del proyecto, asegurando que pueda operar de manera estable bajo condiciones de alta carga o en entornos de producción.

\section{Avances del 2026-01-14}

\subsection*{Añadido de efecto overlay con tintes en la visualización}
Se incorporó un efecto de superposición con tintes de color en el visualizador de partituras, lo que mejora la capacidad para distinguir diferentes elementos y resaltar diferencias mediante coloración selectiva. Esta solución visual es especialmente útil para la comparación detallada de documentos musicales.

\subsection*{Incorporación de zoom automático}
Se implementó un sistema de zoom automático en el visualizador, que ajusta la escala de la partitura para optimizar su presentación en pantalla. Este mecanismo facilita la inspección visual sin que el usuario necesite realizar ajustes manuales constantes.

\subsection*{Visualización por superposición añadida}
Se añadió un nuevo modo de visualización por superposición para las partituras, lo que permite observar simultáneamente diferentes versiones o estados del documento musical combinados en una sola vista. Esta modalidad aporta una nueva perspectiva analítica para el usuario.

\subsection*{Desarrollo del visualizador interno para partituras}
Se completó la implementación de un visualizador interno capaz de mostrar partituras dentro de la aplicación, eliminando la dependencia de herramientas externas. Esta integración mejora la ergonomía del sistema y agiliza el flujo de trabajo en la transcripción y análisis musical.

\section{Avances del 2026-01-16}

\subsection*{Integración y sincronización del repositorio principal}
Durante esta fecha se realizaron dos integraciones importantes para mantener el proyecto actualizado con la rama principal del repositorio remoto. Estas fusiones garantizaron que el desarrollo local incorporara los últimos cambios externos, lo cual es fundamental para evitar conflictos y asegurar la coherencia del código a largo plazo.

\subsection*{Corrección del scroll en el modo comparación}
Se corrigió un problema relacionado con la navegación mediante scroll en el modo de comparación de partituras. Este ajuste mejora la usabilidad y la experiencia del usuario al analizar diferencias entre dos documentos musicales, asegurando que el desplazamiento sea fluido y preciso.

\subsection*{Implementación de reducción individual de la imagen roja en modo superposición}
Se desarrolló una función para reducir individualmente la imagen codificada en rojo cuando se utiliza el modo de superposición de partituras. Esta mejora permite un ajuste más fino en la representación visual, facilitando la distinción y comparación entre capas superpuestas de información musical.

\subsection*{Interacción mediante ratón para mover la partitura nueva sobre la antigua}
Se añadió una interacción gráfica que posibilita al usuario desplazar la partitura nueva sobre la antigua utilizando el ratón. Esta funcionalidad incrementa la interactividad del visualizador, otorgando mayor control manual en la comparación y superposición de documentos.

\section{Avances del 2026-01-17}

\subsection*{Actualización automática del documento memoria}
Durante esta fecha se realizaron acciones orientadas a automatizar la actualización de la memoria del proyecto, facilitando la integración continua y asegurando que los documentos que describen el avance del trabajo se mantengan al día con los últimos cambios implementados. Esta mejora técnica contribuye a agilizar el proceso de generación de documentación y reduce posibles errores derivados de desincronizaciones manuales.

\subsection*{Incorporación de cambios previos en la memoria}
Se añadieron al archivo de memoria del proyecto las modificaciones correspondientes a días anteriores, concretamente las del 14, 15 y 16 de enero, consolidando la documentación de todos los avances relevantes en el período. Este paso garantiza una trazabilidad completa del desarrollo y facilita la revisión técnica posterior.

\section{Avances del 2026-01-20}

\subsection*{Integración completa de transcripción a imágenes musicales}
En esta etapa se realizó la fusión de la rama dedicada a la conversión completa de partituras transcritas en imágenes, consolidando así el proceso que transforma la información musical en formatos visuales. Esta integración representa un paso fundamental para ofrecer al usuario una representación gráfica precisa y manipulable de las partituras obtenidas mediante la transcripción.

\subsection*{Incorporación de barra de progreso tipo temporizador durante la conversión}
Se añadió una barra de progreso que actúa como temporizador visual durante el proceso de conversión, con el fin de mejorar la experiencia del usuario manteniéndolo informado y atento mientras se lleva a cabo la transcripción. Debido a la naturaleza no determinista de la duración del proceso, la barra no se diseñó para reflejar un progreso exacto, sino como un elemento de entretenimiento y feedback visual. Adicionalmente, se corrigió el controlador asociado para que, una vez finalizada la transcripción, muestre automáticamente la última partitura procesada, garantizando que el usuario tenga acceso inmediato al resultado más reciente.

\subsection*{Añadidos botones para la navegación entre páginas de la partitura}
Se implementaron controles específicos que permiten al usuario cambiar entre las distintas páginas de una partitura cuando esta se compone de múltiples hojas. Esta funcionalidad incrementa notablemente la usabilidad del visualizador, dotándolo de la capacidad de recorrer con facilidad todo el contenido musical, haciendo posible así una inspección detallada y cómoda de cada página de la partitura.

\section{Avances del 2026-01-20}

\subsection*{Integración de ramas y consolidación del desarrollo}
En esta fase del desarrollo, se realizó la integración de la rama principal del repositorio remoto con la rama local del proyecto. Esta operación permitió consolidar los avances más recientes y asegurar que el entorno de desarrollo se encontrase sincronizado con la última versión del código fuente disponible en el control de versiones. La fusión no solo facilita la continuidad del trabajo en el proyecto, sino que también minimiza la aparición de conflictos futuros al mantener una base de código homogénea. Este tipo de integraciones periódicas es esencial para coordinar el trabajo en proyectos colaborativos de ingeniería informática y asegurar la coherencia e integridad del sistema en desarrollo.
